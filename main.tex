% # 公立はこだて未来大学・修士論文テンプレートファイル(unicode)
%
% ## 改訂履歴:
% - 2019/12/01 初版 作成者:三上貞芳
% - 2020/01/12 V1.2 作成者:三上貞芳 英字綴訂正
%
% ## 論文作成の手順
%
% 1. 以下のtexファイルを作成してください
% - cover.tex           氏名・タイトル等の表紙情報
% - eabstract.tex       英語アブストラクト
% - jabstract.tex       日本語アブストラクト
% - chapterX.tex        本文第X章
% - publications.tex    発表・採録等の実績(確定分も含む)
% - acknowledgment.tex  謝辞
% - bibliography    .tex    参考文献
%
% 2. このテンプレートの「TODO: 本文」以下に,作成した章に対応する\input{chapterX.tex}文を追記してください(Xは章番号)
%
% 3. このテンプレートとfunstyle_master.texと合わせてuplatex環境でコンパイルし,PDFを作成します.
%

\documentclass[uplatex, a4paper, report, 11pt, oneside]{jsbook}

% packages
\usepackage[utf8]{inputenc}
\usepackage[dvipdfmx]{graphicx}
\usepackage{lmodern}             % use latin modern font
\usepackage{amsmath,amssymb,amsthm}
\usepackage{url}
\usepackage{here}
\usepackage{layout}

\newcommand{\mason}{\protect{\sc Mason}}
% 未来大書式設定
% \input{funstyle_master.tex}

% TODO: タイトル・著者等の情報
% TODO: 論文題目等の情報を以下に記入

\newcommand{\jdoctitle}{修士論文}
\newcommand{\edoctitle}{Master's Thesis}
\newcommand{\jtitle}{人間とAIの協調:ChatGPTを用いたMinecraftBOTによる作業支援}  % 修論の題名
\newcommand{\etitle}{Human and AI Co-operation: Minecraft Server and Bots}   % 論文題目(英)
\newcommand{\jauthor}{工藤 光矢}      % 著者名(日)
\newcommand{\eauthor}{Koya Kudo} % 著者名(英)
\newcommand{\jadvisor}{フランク イアン}   % 指導教員名(日)
\newcommand{\eadvisor}{Ian Frank}  % 指導教員名(英)
\newcommand{\jdate}{2024年2月XX日}  % 論文提出日   (日)
\newcommand{\edate}{February XX, 2024}  % 論文提出年月 (英)
\newcommand{\jkeywords}{人工知能, 大規模言語モデル, Minecraft} % キーワード(日)
\newcommand{\ekeywords}{Artificial Intelligence, LLM, Minecraft}   % キーワード(英)
\newcommand{\eshorttitle}{Human and AI Co-operation: Minecraft Server and Bots}    % 短縮英題題名(おおよそ8 words以内)
\newcommand{\jaffiliation}{知能情報科学領域}    % 領域名(日)
\newcommand{\eaffiliation}{Intelligent Information Science Field}    % 領域名(英)


% TODO: 英語アブストラクト
% TODO: 英文アブストラクトを以下の{}内に記述(以下はダミーテキスト)
\newcommand{\eabstract}{

Games have long been important as a testbed and basic research for improving the performance of artificial intelligence and algorithms, and in recent years AI has surpassed human scores in complex games such as Go. 
However, even though AI has overwhelmed humans, game research is still valuable as a testbed and basic research, and new approaches, such as human-AI cooperation, are being seen. 
This research aims to explore the boundaries and possibilities of cooperation and collaboration between humans and AI in the context of ever-changing artificial intelligence technologies, and has developed a bot "Mason" (Minecraft AI Support - Objectives and Novelty) that works with players in the virtual 3D game "Minecraft" using the ChatGPT API. 
which can suggest the next task suitable for the current situation by observing the player's information and conversations with the player. 
It can also generate and customise structures such as houses and furniture in the gameplay world by issuing Minecraft fill commands according to user instructions. 
When these functions were tested, the LLM's spatial grasp and response performance suggested the boundaries of the current cooperation and collaboration between humans and AI. 
The ability to embed bot functions in virtual space is rapidly expanding, and these results can be seen as an early stage in the activity of embedding bots in virtual space. 
Minecraft also has applications as an educational tool, and Mason's potential for human-AI cooperation is that Mason can introduce students to the basics of virtual worlds Mason could be useful for introducing students to the basics of virtual worlds, developing their creativity and communication skills, and familiarising them with LLMs. 
Future plans include the need to improve the prompts to allow for more complex structures to be generated and interacted with, and for stronger inspiration and competence; the Mason framework is publicly available and could be adapted by others to explore the creative The possibility of exploring the perspectives enabled by the development of generative AI is also conceivable.

}

% TODO: 日本語アブストラクト
% TODO: 日本語アブストラクトを以下の{}内に記述(以下はダミーテキスト)
\newcommand{\jabstract}{

    ゲームは人工知能やアルゴリズムの性能向上のための実験台や基礎研究として古くから重要視されており,近年では囲碁などの複雑なゲームにおいてAIが人間のスコアを超越するようになった.
    しかし,AIが人間を圧倒してもなお,ゲーム研究は実験台や基礎研究として価値のあるものであり,人間とAIの協力関係と言った新たなアプローチが見られるようになっている.
    本研究では変化し続ける人工知能技術の中で,人間とAIとの協調・協力関係の境界と可能性を探ることを目的としており,仮想3Dゲーム「Minecraft」において,ChatGPT APIを活用してプレイヤーと連携するbot "Mason"(Minecraft AI Support - Objectives and Novelty)を作成した.
    Masonはプレイヤーとの会話やプレイヤーの情報を観察することで,現在の状況に適した次のタスクを提案することができる.また,ユーザーの指示に従ってMinecraftのfillコマンドを発行することで,ゲームプレイの世界に家や家具などの構造物を生成し,カスタマイズすることもできる.これらの機能をテストしたところ,LLMの空間把握能力や応答性能から,現状の人間とAIとの協調・協力関係の境界が示唆された.ボット機能を仮想空間に埋め込む能力は急速に拡大しており,この結果はbotを仮想空間に埋め込む活動の初期段階とみなすことが出来る.Minecraftは教育ツールとしての用途もあり,Masonの人間とAIの協調の可能性として,Masonは生徒に仮想世界の基礎を紹介したり,創造性やコミュニケーション力を鍛え,LLMに慣れ親しんでもらうのに役立つと考えられる.今後の予定としては,プロンプトの改善によってより複雑な構造物を生成や対話を可能にし,より強いインスピレーションや能力を得られるようにする必要があると考えられる.Masonフレームワークは公開しており,他の人が適応して,生成AIの発展によって可能になった創造的な展望を探求可能性も考えられる.

}

% page size
\textheight     = 22.6truecm
\textwidth      = 14.7truecm
\oddsidemargin  = 0.6truecm

% header and footer
\usepackage{fancyhdr}
\pagestyle{fancy}
\setlength{\footskip}{16pt}
\fancyhf{}
\renewcommand{\chaptermark}[1]{\markboth{\thechapter.\ #1}{}}
\rhead{\leftmark}
\renewcommand{\headrulewidth}{0pt}
\cfoot{\thepage}
\lfoot{~~ \\Master's thesis, Future University Hakodate}
\lhead{\eshorttitle}

%-------------------------------------
\begin{document}

\thispagestyle{empty}
\vspace*{1.5truemm}
\begin{center}
    \LARGE\bfseries
    修士論文
\end{center}
\vspace*{2truemm}
\begin{center}
    \LARGE\bfseries\jtitle
\end{center}
\vspace*{2em}
\begin{center}
    \large\bfseries 公立はこだて未来大学大学院~~システム情報科学研究科\par%
    \jaffiliation
\end{center}
\vspace*{1em}
\begin{center}
    \Large\bfseries\jauthor
\end{center}
\vspace*{1em}
\begin{center}
    \large 指導教員~~~~\jadvisor\par
    \vspace{0.5em}
    \large 提出日~~~~\jdate
\end{center}
\vspace*{3em}
\begin{center}
\textbf{\Large Master's Thesis}\par
\vspace*{2em}
\textbf{\Large \etitle}\par
\vspace*{1em}
{\normalsize by}\par
\vspace*{1em}
{\large \eauthor}\par
\vspace*{1.5em}
Graduate School of Systems Information Science, Future University Hakodate \par
\eaffiliation

% \vspace*{0.5em}
\normalsize Supervisor: \quad \eadvisor \par
\vspace*{2em}
Submitted on \edate
\end{center}
%\vspace*{\fill}

% 英語アブストラクト作成
\newpage
\thispagestyle{empty}
\vspace*{30truemm}
\noindent
\textbf{Abstract--}~
\eabstract

\vspace*{1em}
\noindent
\textbf{Keywords:}~ 
\ekeywords

% 日本語アブストラクト作成
\newpage
\thispagestyle{empty}
\vspace*{30truemm}
\noindent
\textgt{概~要:}~
\jabstract

\vspace*{1em}
\noindent
\textgt{キーワード:}~ 
\jkeywords


% 目次
\tableofcontents
\thispagestyle{empty}

% ページ番号初期化
\setcounter{page}{0}

% TODO: 本文
\chapter{序論}	% TODO: 章題を記入.題は任意.
\thispagestyle{plain}   % chapterの直後に必ず指定

%TODO: 章の内容を記入.以下はサンプル.

\section{はじめに}

\section{目的}

\section{章構成}
\chapter{関連研究}	
\thispagestyle{plain}   % chapterの直後に必ず指定

本章では,本研究に関する,人間とAIの協調,Minecraftの教育性,自動生成,Minecraftのエージェント等の研究,文献について述べる.

\section{人間とAIの協調に関する研究}
この節では人間との協調を目的としたAIに関する関連研究について述べる.

\subsection{Maia}
Kleinbergらは,人間らしい手を打つニューラルネットワークチェスエンジンの“Maia”を開発した\cite{bib:maia}.
Maiaはオンラインでチェスを行うプレイヤーの棋譜を,強さ別に9つのレベルに分けて学習しており,レベル別にプレイヤーがどのようなチェスの指し方をしているかを学べるようになったと述べている.
また,低いレベルと高いレベルの学習結果の違いから,未熟なチェスプレイヤーがどのような間違いを犯すかを「チェス学習ツール」として予測することが可能である.
Kleinbergは,人間に教えたり助けたりするAIの研究によって,人間ともっとうまく交流できるかもしれない,あるいは交渉すらできるかもしれないと述べている.


\section{Minecraftの教育性に関する研究}
この節ではMinecraftの教育性に関する関連研究について述べる.

\subsection{学習ツールとしてのMinecraft}
Cipolloneらは高校の文学の授業でMinecraftを使用し,文学の授業を受けた生徒らはMinecraftを録画する形で映画を3本作成した\cite{bib:minecraft_creative tool}.
Cipolloneらは作成された映画をプロットとキャラクターの観点で検討したところ,Minecraftを用いたことで生徒が創造性を発揮し,概念を理解しやすくなったと述べている.
また,Minecraftは他の方法ではコストがかかったり不可能だったりするような作品を生徒が作れるようになるツールであると述べている.
しかし,ゲーム文化と,正式な学校教育の文化とではギャップが大きいため,Minecraftのような教育ゲームを導入するためには,ゲーム文化をもっと受け入れられるような考えが必要かもしれないと提案している.


\section{自動生成に関する研究}
この節では自動生成に関する関連研究について述べる.

\subsection{ARLPCG}
Gisslénらは,手続き型コンテンツ生成の新たなアプローチとして,Adversarial Reinforcement Learning for Procedural Content Generation(ARLPCG)を提案した\cite{bib:arlpcg}.
手続き型コンテンツ生成の一般的なアプローチとしては,異なる環境を手続き的に生成し,学習したエージェントの汎用性を高める方法が挙げられるが,ARLPCGでは代わりに,PCGエージェント(Generator)と攻略エージェント(Solver)からなる敵対的モデルを用いている.
GeneratorはSolverのパフォーマンスに基づいて報酬信号を受け取っており,攻略不可能ではないが挑戦的なコンテンツ生成が可能になったと述べている.
また,モデルの制御を行うためにGeneratorに補助入力を使用し,検証として3DCGのアクションゲームとレースゲームの2つを生成したところ,ARLPCGの解答率が大幅に向上し,補助入力によって難易度がある程度調整可能になったと述べている.

\subsection{GANcraft}\label{sec:gancraft}
Haoらは,Minecraftのような3Dのブロック世界からリアルな地形画像を生成する,GANcraftを開発した\cite{bib:gancraft}.
GANcraftは教師なしのニューラルレンダリングフレームワークである.
本手法では,ブロック世界を入力とし,各ブロックには土,草,木,砂,水などのラベルが割り当てられる.
また,ブロックの世界を連続的な体積関数として表現し,教師データがない場合でも,任意の視点からなリアルな地形画像をレンダリングできるようにモデルを学習できると述べている.
GANcraftは,その他のフレームワークと比較して有効性を示しており,長年の経験が必要とされる複雑な風景の3Dモデリングを簡単に行うことができると述べている.

\section{Minecraftのエージェントに関する研究}
この節ではMinecraftのエージェントに関する関連研究について述べる.

\subsection{CraftAssist}
Grayらは,Minecraftで人間のプレイヤーと協力できるAIアシスタントを実装するためのプラットフォームとして,CraftAssistを開発した\cite{bib:craft_assist}.
CraftAssistのアシスタントはブロックの配置や破壊,人間のプレイヤーとのチャットが可能であり,また,言語,知覚,記憶,身体的動作を組み合わせることで,家の建設などの複雑なタスクを実行できる.
このような研究は,実世界で人間とより良く対話し協力できるAIアシスタントを実現する手助けとなる可能性があり,AIアシスタントとの対話を通じて新しい概念とスキルを積極的に学び取ることができるかもしれないと述べている.

\subsection{Voyager}
Wangらは,Minecraftで大規模言語モデルを使用し,人間の介入なしに継続的に世界を探索するエージェントのVoyagerを開発した\cite{bib:voyager}.
VoyagerはAPIを介して自分の所持アイテムや近くの地形などのゲーム情報を読み取ることができ,GPT-4を用いて,情報をもとに短期的な目標を設定したり,目標を達成するために必要なコードを生成したりすることが可能である.
Voyagerはほかのモデルと比較して広い範囲の地形を探索し,多くのアイテムを製作することができたと述べている.
またWIRED誌では``このような方法で言語モデルを活用すれば,オフィスで多くの日常的な作業を自動化できるかもしれない.それがVoyagerがもたらした最大の経済効果であろう''と述べられている.

\chapter{環境}	
\thispagestyle{plain}   % chapterの直後に必ず指定

本章では,本研究のゲーム環境であるMinecraftやテスト環境のFUN Minecraft Serverについて述べる.

\section{Minecraftとは}\label{sec:minecraft}
Minecraft\cite{bib:Minecraft}とは,創造性やサバイバル生活に焦点を当てた,サンドボックス型のゲームである.Minecraftの世界は図\ref{fig:mc_world}のように,3D空間で水平方向にほぼ無限に広がっており,木,石,土,水など様々な種類の一定の大きさの立方体ブロックで構成され,その世界にはドット絵調の動植物が存在する.プレイヤーは主に以下の二つのモードでゲームを楽しむことができる.\\

\textbf{サバイバルモード} 冒険と生活を主体としたモードである.体力や空腹度の概念が存在するため,プレイヤーは死を免れるために狩猟,採集,農業,牧畜などで食料を収集し,ブロックの収集や集めたアイテムの加工 (クラフト)を駆使して敵から逃れるための安全な住居を作らなければならない.\\

\textbf{クリエイティブモード} ブロックの組み立てに重きをおいた,創造性や実験を主体としたモードである.サバイバルモードと比較して,体力や空腹度の概念がなく,組み立てがしやすいように自由に飛行することが可能である.また,すべてのブロックを無限に配置でき,瞬時に破壊することが可能である.したがって,サバイバルモードでは制作困難な巨大建築や,複雑な回路の作成を楽しむことが可能となっている.\\
\begin{figure}[H]
    \centering
    \includegraphics[width=0.7\textwidth]{fig/minecraft_world.png}
    \caption{Minecraftの世界}
    \label{fig:mc_world}
\end{figure}

\section{Minecraftの特徴}\label{sec:minecraft_feature}
Minecraftには,以下のような学術的に有利な特徴がある.\\

\textbf{認知度} Minecraftは2011年にPCで正式版がリリースして以来,iOSやAndroidなどのスマートフォン機種や,Xbox One,Nintendo Switchなどの家庭用ゲーム機,その他様々なプラットフォームに移植された.
また,2023年10月にはMinecraftの累計売り上げ本数が3億本を突破したことが公式ライブで発表され,世界で最も売れたインディーゲームとなった\cite{bib:minecraft_news}.
このように対応するプラットフォームが多く認知度が高いことから,ゲームのルールや特徴を知っている人が多いと予想できるため,評価実験やアンケートなどの客観的尺度が必要になった場合には遂行しやすくなることが考えられる.\\
%TODO: いつだったかの発表の好きな実況ゲームランキング入れる

\textbf{ゲーム環境} Minecraftは特定の目標がなく,オープンワールドであるため,AIの行動の可能性を広げることが可能である.
また,場合によってはブロックの設置により,逆に行動を制限させ,柔軟にAIの学習環境を設定することが可能である.\\

\textbf{拡張性} Minecraftは拡張機能 (MOD) やツールの種類が多く,Minecraft最大手のMOD公開サイトとなっているCurseForge\cite{bib:curseforge}では,2024年1月現在で約153,204のMODプロジェクトが公開されている.
多くのMODが開発されている理由として,Minecraft EULA\cite{bib:eula}ではMinecraftのMOD (プログラムの修正,ツール,プラグイン含む)の追加を許可しており,開発のためのコミュニティも活発であることが挙げられる.
したがって,拡張性が高いため研究に合わせてゲーム性を保ったまま自由にMinecraftをカスタムすることが可能である.\\

\textbf{教育的価値} サバイバルや建築を楽しむことが可能であるゲーム性から,問題解決力や創造性を養うことが可能である.
その教育性からMinecraft公式は,Minecraft Education (教育版マインクラフト)\cite{bib:minecraft_edu}をリリースしている. \\

\section{FUN Minecraft Serverについて}
FUN Minecraft Server\cite{bib:fun_minecraft_server}は,公立はこだて未来大学の協力と支援のもとで成り立っている公認のサーバーである.
複数の研究室と多くの有志メンバーから「Minecraftを用いた数多くの研究に繋がる可能性」への共感を得て,実現が可能となった.
FUN Minecraft Serverは,MinecraftのJava版のアカウントを取得しており,公立はこだて未来大学に所属している人であれば参加することが可能である.
FUN Minecraft Serverプロジェクトの目的は,FUN Minecraft Serverのオンライン環境を通して,公立はこだて未来大学に所属する学生や教員が研究のアイデアの着想を得ることである.
海外の大学では,同様の目的でMinecraftの大学公式サーバーを運用する事例が複数あり,効果的な成果があったと報告されている\cite{bib:cambridge_minecraft_server, bib:texas_minecraft_server}.
国内の大学におけるMinecraftの公式サーバー運用に関する事例は,FUN Minecraft Serverと同時期に設立されたプロジェクトが確認された\cite{bib:tducraft}.しかし,そのような事例は依然として限られており,その実施は比較的稀であると言える.
したがって,FUN Minecraft Serverが国内の大学の公式サーバのモデルケースになる可能性が期待される.

実際にFUN Minecraft Serverは様々な研究の環境として利用されている.
本研究で作成した{\mason}もFUN Minecraft Serverを動作テストやアンケートを行うための環境として用いている.
過去には,Minecraftで作成したビデオを使用して,小学生向けの教材を作成した研究が存在していた\cite{bib:fun_minecraft_server_research}.
また,現行の研究では,BEVI-J評価ツールを使用して,FUN Minecraft Serverに参加した学生の人物像についての研究\cite{bib:fun_minecraft_server_research}や,運営側でサーバーにPlan,CoreProtect,DiscordSRVなどのプラグインを導入し,コミュニティをモニタリング・管理し,ゲームプレイデータを収集・分析する研究などが行われている\cite{bib:fun_minecraft_server_research,bib:fun_minecraft_server_research2}.
\chapter{開発}	% TODO: 章題を記入.題は任意.
\thispagestyle{plain}   % chapterの直後に必ず指定

%TODO: 章の内容を記入.以下はサンプル.
本研究では,\ref{chap:minecraft}章で述べたMinecraftの特性に焦点を当て,Minecraftの中でLLMを搭載したBOTを作成し,そのBOTと人間が対話を行ったり,共同作業を行ったりすることで,人間とAIの協力・協調関係の検証を行う.

本章では作成したBOTについて解説する.
システムの全体像は図\ref{fig:system}のとおりである.
プレイヤーはBOTログイン用webアプリケーションを使用して,BOTをMinecraftのマルチサーバーに参加させることが可能である.
その後プレイヤーは,BOTが参加したマルチサーバーと同じサーバーにログインしMinecraftのチャットの機能(デフォルトでTキー)を用いてBOTと対話することができる.

マルチサーバーにログインしたBOTはプレイヤーからのチャットの情報をLLMに送り,LLMからの返答をもとにプレイヤーへ返答や行動を行うことが可能である.
詳細な機能については\ref{sec:webapp}節から\ref{sec:build_mode}節で解説する.

\begin{figure}[H]
    \centering
    \includegraphics[width=0.8\textwidth]{fig/my_system.PNG}
    \caption{システムの全体像}
    \label{fig:system}
\end{figure}

\section{Webアプリケーション}\label{sec:webapp}
人間とAIの協力・協調を主目的とする以上,ユーザビリティについて追求する必要があると考えたため,BOTをログインさせるためのWebアプリケーションを作成した.
従来までの方法でBOTを動かすためには,プログラミング環境構築のスキルを必要とするが,このアプリケーションではマイクラのマルチにログインする時と同様の情報を入力することで,簡単にBOTをログインさせることが可能となった.

\begin{figure}[H]
    \centering
    \includegraphics[width=0.5\textwidth]{fig/web_app.jpg}
    \caption{Webアプリケーションのホーム画像}
    \label{fig:web_app}
\end{figure}

\section{行動}\label{sec:act}
BOTは事前に登録されている行動を実行することが可能である.
BOTの各行動はMineflayer API\cite{bib:Mineflayer}を用いて実装した.
指定位置への移動,所持アイテムをチャット,アイテムの出し入れ,畑に種を植える,指定アイテムを収集する行動などを実行することができる.
また,\ref{sec:build_mode}節のビルドモードも行動として実行できる.

\section{ChatGPTによる応答}\label{sec:gpt_res}
BOTはLLMを搭載したことで自然な返答が可能となっている.
LLMはChatGPTやPaLM,ローカル環境で動作するDollyなどいくつかのモデルを検討し,実行速度,会話の自然さや機能面の観点から自己検証を行ったところChatGPTが良かったことから,ChatGPT(gpt-3.5-turbo-0613)を用いた.

初期プロンプトはVoyager\cite{bib:Voyager}のものを一部参考し,内容としてはChatGPTの役割,各会話のラウンドで与える情報,回答の際に守るべきルールを記載した.
また,ChatGPT APIのFunction Calling機能を用いているので,\ref{sec:act}節の行動を,必要に応じて実行することが可能である.

\section{ビルドモード}\label{sec:build_mode}
BOTは``建築したい''などのチャットを受け取るとビルドモードに移行し,Minecraftの/fillコマンドを用いてChatGPTがプロンプトに沿った構造物を生成することが可能である.

初期プロンプトには,/fillコマンドで構造物を生成すること,/fillコマンドの説明,3次元座標の概念の説明,よく建築で使うブロックの種類,/fillコマンドの使用例を記述した.

\chapter{実験と結果}	
\thispagestyle{plain}   % chapterの直後に必ず指定

\section{タスクの提示や会話機能の動作テスト}\label{sec:gpt_res_test}
この節では\ref{sec:gpt_res}節で開発した{\mason}のサバイバルモードのプレイヤー向けの対話機能について,目標を設定し動作テストの結果を述べる.

今回は{\mason}に表\ref{tab:goal_and_action}の目標や使用できる行動を設定し,{\mason}との対話機能を使用した.

\begin{table}[H]
    \caption{実験1の設定}\label{tab:goal_and_action}
    \centering
    \begin{tabular}{ll}
        \hline \hline
        目標 & ダイヤモンドを集める. \\
        \hline
        許可した行動 & mine\_block \\
          & craft\_item \\
        \hline
    \end{tabular}
\end{table}

目標への最序盤での対話の結果を図\ref{fig:first_task}に示す.
図\ref{fig:first_task}より,{\mason}はプレイヤーが森バイオームに位置し,何も装備していないことなどからゲームの序盤であることを推論し,Minecraftの最も基本的な資源の木を集めるべきであるとタスクを提示していることが確認できた.
また,``木を集める''というタスクは{\mason}も行うことが可能なタスクであるためこの作業を手伝うかの提案を行っていることが確認できた.

\begin{figure}[H]
    \centering
    \includegraphics[width=0.95\textwidth]{fig/first_task.PNG}
    \caption{序盤のタスクの提示}
    \label{fig:first_task}
\end{figure}

木を切るタスクを手伝ってもらうために``y''で返答することで,図\ref{fig:task_help}のようにプレイヤーに提示したタスクを{\mason}も行うことが可能であることを確認できた.
\begin{figure}[H]
    \centering
    \includegraphics[width=0.95\textwidth]{fig/task_help.PNG}
    \caption{木を切る作業の支援}
    \label{fig:task_help}
\end{figure}

{\mason}はプレイヤーから話しかけられなくても定期的にタスクを提示することができ,木を切った後に図\ref{fig:wooden_pickaxe}のように木のピッケルを作るべきだというタスクを提示している.
木のピッケルの作り方がわからないプレイヤーが作り方を質問した場合,その質問に回答することが可能だった.
\begin{figure}[H]
    \centering
    \includegraphics[width=0.95\textwidth]{fig/wooden_pickaxe.png}
    \caption{木を切る作業の支援}
    \label{fig:wooden_pickaxe}
\end{figure}

また,図\ref{fig:diamond_question}より,プレイヤーがダイヤモンドを取得した際にダイヤモンドを使用し,どのようなアイテムを作成可能かといった質問にも回答することが可能であることを確認できた.
よって,目標の達成とは関係ないプレイヤーからの質問に対して,MASONは回答することが可能であることが確認できた.
\begin{figure}[H]
    \centering
    \includegraphics[width=0.95\textwidth]{fig/diamond_question.PNG}
    \caption{目標と関係ない質問をしたときの対話}
    \label{fig:diamond_question}
\end{figure}

さらに図\ref{fig:server_version}のように,元の大規模言語モデルが学習していないと考えられる質問にも答えることが可能である.
\begin{figure}[H]
    \centering
    \includegraphics[width=0.95\textwidth]{fig/server_version.PNG}
    \caption{未知のデータに対する回答}
    \label{fig:server_version}
\end{figure}

目標の最終盤での対話の結果を図\ref{fig:final_task}に示す.
図を見ると,{\mason}はプレイヤー鉄のピッケルを持っていることからダイヤを採掘するタスクを提示していることを確認できた.

\begin{figure}[H]
    \centering
    \includegraphics[width=0.95\textwidth]{fig/final_task.png}
    \caption{ダイヤモンド採掘の提示}
    \label{fig:final_task}
\end{figure}

ただし,稀に次に提示するタスクが適切ではない(例えばプレイヤーが石を採掘出来ていないにも関わらず石のピッケルの作り方を提示した)ことや,質問に対する回答が間違っている結果が見られた.

\section{構造物自動生成の実験}\label{sec:build_mode_generate}
この節では,\ref{sec:build_mode}節にて解説したクリエイティブモード向け構造物自動生成機能がどのような構造物を生成可能であるかを,モデルやプロンプトの設定を踏まえて紹介する.

\subsection{実験1}\label{sec:ex1}
実験1では,{\mason}の自動生成機能において,詳細度の異なる2つのプロンプトをそれぞれ送信し,出力結果の比較を行った.今回の実験の設定内容は表\ref{tab:setting1}に記す.
今回のテストケースとして駅の構造を生成した.
なお,大規模言語モデルによる出力結果は一定ではないため,表\ref{tab:setting1}の通り,それぞれのプロンプトを3度送信した.
その中から,それぞれ最も優れた結果を図\ref{fig:station1}に示す.
\begin{table}[H]
    \caption{実験1の設定}\label{tab:setting1}
    \centering
    \begin{tabular}{ll}
        \hline \hline
        モデル & gpt-3.5-turbo \\
        \hline
        左図プロンプト & train station \\
        \hline
        右図プロンプト & Build a train station.I want the tracks to use rails. \\
          & And build the platform well. \\
        \hline
        試行回数 & 3回 \\
        \hline
    \end{tabular}
\end{table}

\begin{figure}[H]
    \centering
    \includegraphics[width=0.95\textwidth]{fig/train_station1.PNG}
    \caption{実験1の生成結果}
    \label{fig:station1}
\end{figure}

図\ref{fig:station1}右図の出力結果を見ると駅の外装のような構造物を作成していることが示唆される.
しかし,左図構造物生成時のログを参照すると,下部の石レンガはプラットフォーム,上部の板材,ダークオークのフェンス,ガラスパネルはチケットカウンター,ベンチ,標識であると記載されているため,この出力結果は駅の外装ではなくモデルがあまり駅のプラットフォームについて理解せず生成したものであると言える.
右図では,ログを参照したところ板材と石レンガでプラットフォームを作り,それに沿うようにレールを敷いていると記載されているため,プロンプトによって出力改善されたことが分かる.

\subsection{実験2}\label{sec:ex2}
実験2では,{\mason}の自動生成機能において,詳細度の異なる2つのプロンプトをそれぞれ送信し,出力結果の比較を行った.今回の実験の設定内容は表\ref{tab:setting2}に示す.
実験1と同様に今回のテストケースとして駅の構造を生成した.
なお,大規模言語モデルによる出力結果は一定ではないため,表\ref{tab:setting2}の通り,それぞれのプロンプトを3度送信した.
その中から,それぞれ最も優れた結果を図\ref{fig:station2}に示す.
\begin{table}[H]
    \caption{実験2の設定}\label{tab:setting2}
    \centering
    \begin{tabular}{ll}
        \hline \hline
        モデル & gpt-4 \\
        \hline
        左図プロンプト & train station \\
        \hline
        右図プロンプト & Build a train station.I want the tracks to use rails. \\
          & And build the platform well. \\
        \hline
        試行回数 & 3回 \\
        \hline
    \end{tabular}
\end{table}

\begin{figure}[H]
    \centering
    \includegraphics[width=0.95\textwidth]{fig/train_station2.PNG}
    \caption{実験2の生成結果}
    \label{fig:station2}
\end{figure}

\subsection{実験3}\label{sec:ex3}
実験3では,{\mason}の自動生成機能において,プロンプトによる構造物の生成を行い,その後追加のプロンプトを送信し,出力結果がどのように変化するかを検証した.今回の実験の設定内容は表\ref{tab:setting3}に示す.
今回のテストケースとしてまず家を作成するプロンプト(表\ref{tab:setting3}プロンプト1)を送信し,その後窓を追加するプロンプト(表\ref{tab:setting3}プロンプト2)を送信した.
なお,大規模言語モデルによる出力結果は一定ではないため,表\ref{tab:setting3}の通り,プロンプトを3度送信し,その中から,最も優れた結果を図\ref{fig:add_window}に示す.

\begin{table}[H]
    \caption{実験3の設定}\label{tab:setting3}
    \centering
    \begin{tabular}{ll}
        \hline \hline
        モデル & gpt-4-1106-preview (GPT-4-Turbo) \\
        \hline
        プロンプト1 & please make house. \\
        \hline
        プロンプト2 & please add window. \\
        \hline
        試行回数 & 3回 \\
        \hline
    \end{tabular}
\end{table}

\begin{figure}[H]
    \centering
    \includegraphics[width=0.95\textwidth]{fig/add_window.PNG}
    \caption{実験3の生成結果}
    \label{fig:add_window}
\end{figure}

\subsection{実験4}\label{sec:ex4}
実験4では,{\mason}の自動生成機能において,プロンプトによる構造物の生成を行い,その後追加のプロンプトを送信し,出力結果がどのように変化するかを検証した.今回の実験の設定内容は表\ref{tab:setting4}に示す.
実験3と同様にテストケースとしてまず家を作成するプロンプト(表\ref{tab:setting4}プロンプト1)を送信し,その後窓を追加するプロンプト(表\ref{tab:setting4}プロンプト2)を送信した.
なお,大規模言語モデルによる出力結果は一定ではないため,表\ref{tab:setting4}の通り,プロンプトを3度送信し,その中から,最も優れた結果を図\ref{fig:add_window2}に示す.

\begin{table}[H]
    \caption{実験4の設定}\label{tab:setting4}
    \centering
    \begin{tabular}{ll}
        \hline \hline
        モデル & gpt-3.5-turbo-1106ファインチューニング済みモデル \\
        \hline
        プロンプト1 & please make house. \\
        \hline
        プロンプト2 & please add window. \\
        \hline
        試行回数 & 3回 \\
        \hline
    \end{tabular}
\end{table}

\begin{figure}[H]
    \centering
    \includegraphics[width=0.95\textwidth]{fig/add_window2.PNG}
    \caption{実験3の生成結果}
    \label{fig:add_window2}
\end{figure}

\section{構造物の自動生成に関するアンケートの実施と結果}\label{sec:survey_result}
構造物の自動生成の有用性を検証するためbotのデモとアンケートを実施した.
デモではbotとの会話を始めに体験させ,その後,クリエイティブモード向け構造物自動生成を体験するように促した.
デモ・アンケートは2023年10月16日に開催された,公立はこだて未来大学オープンラボの,イアンフランク研究室のブースにて行われ,計16人の学生がデモを行いアンケートに回答した.

デモを行った後のアンケートの結果について表\ref{tab:answer1}~表\ref{tab:answer4}に示す.

``BOTは現状,人間の作業や創造性を助けることができていると思いますか?''という質問について,``そう思う''を4,``どちらかというとそう思う''を3,``どちらかというとそう思わない''を2,``そう思わない''を1として平均を計算した結果,3.27となった.
\begin{table}[H]
    \centering
    \caption{BOTは現状,人間の作業や創造性を助けることができていると思いますか?}
    \label{tab:answer1}
    \includegraphics[width=0.7\textwidth]{fig/tab1.PNG}
\end{table}

``このBOTを使うことでLLM(ChatGPTなど)の使い方を新しく学ぶことができましたか?''という質問については同様の方法で平均を求めた結果,3.47となった.

\begin{table}[H]
    \centering
    \caption{このBOTを使うことでLLM(ChatGPTなど)の使い方を新しく学ぶことができましたか?}
    \label{tab:answer2}
    \includegraphics[width=0.7\textwidth]{fig/tab2.PNG}
\end{table}


``このBOTを使うことでMinecraftの知識を新しく学ぶことができましたか?''という質問については同様の方法で平均を求めた結果,3.07となった.
%表\ref{tab:answer1}の結果では肯定的な意見が得られているものの,表\ref{tab:answer2}の結果より平均が低いことや\ref{sec:build_mode_generate}節の生成結果から考察すると構造物の自動生成は改善の必要があると考えられる.
\begin{table}[H]
    \centering
    \caption{このBOTを使うことでMinecraftの知識を新しく学ぶことができましたか?}
    \label{tab:answer3}
    \includegraphics[width=0.7\textwidth]{fig/tab3.PNG}
\end{table}

また,``このBOTを使うことで新しい建築のアイデアなどを得ることができましたか?''の質問については同様の方法で平均を求めた結果,3.2となった.
\begin{table}[H]
    \centering
    \caption{このBOTを使うことで新しい建築のアイデアなどを得ることができましたか?}
    \label{tab:answer4}
    \includegraphics[width=0.7\textwidth]{fig/tab4.PNG}
\end{table}

%そう思う~そう思わないの4段階の同じ指標で回答してもらった結果,表\ref{tab:answer2}の結果のみ平均3.47と高かったことから,botによって大規模言語モデルの新しい使い方を学べたことが示唆された.





\chapter{考察}	% TODO: 章題を記入.題は任意.
\thispagestyle{plain}   % chapterの直後に必ず指定

%TODO: 章の内容を記入.以下はサンプル.

\chapter{まとめ・展望}	
\thispagestyle{plain}   % chapterの直後に必ず指定

\section{まとめ}
本研究は,人間とAIの協力・協調関係のを目的とし,Minecraftの中で人間との対話や,作業の支援を行うbot,{\mason}の作成を行った.
{\mason}はタスクの提示やプレイヤーとの会話機能,構造物の自動生成機能などを有しており,それらの動作テストを行いその結果の考察を行った.
考察からはテストレベルでの興味深い結果を得られることが判明したものの実用段階に至るにはまだいくつかの発展が必要であるという課題が示唆された.
一方で大規模言語モデルの発展が著しいため,将来的にはより高精度なモデルが使用可能,ファインチューニング可能になることで,
コミュニケーションスキルや問題解決スキル,創造的なインスピレーションをAIから得られる可能性が考えられた.

\section{展望}
{\mason}は一般的なフレームワークをGithub(\url{https://github.com/koya328/Mason})で公開予定である.
大規模言語モデルがより創造的に進化した場合,他の開発者が{\mason}を使用して可能性を探求することが可能である.
現状での{\mason}の改良点として,サバイバルモード向けの対話機能は,複数の大規模言語モデルの組み合わせにより実現できているため,カスタマイズすることで,より多彩な行動・会話ができるのではないかと考えられる.
例として,複数人のコミュニティで使用するのであれば,社会的でないプレイヤーの発言などを監視する機能や,サーバーのログを収集し,サーバーの最新の情報について質問可能な機能などが考えられる.
構造物の自動生成ではファインチューニングのデータ件数が少なかった.データを増やすことにより,空間把握能力はないものの,人間がイグルーやプラミッドなどの特殊な構造物を作る際,どのようなブロックを使うかなどの情報は学習できるのではないかと考えられる.
% 以下必要に応じてchapterX.texを作成してinput文を記入

% TODO: 謝辞
\pagestyle{plain}
\chapter*{謝辞}
% TODO: 謝辞を以下に記入

本研究において, 数々のご指導を賜りましたIan Frank教授, {\mason}のテスト環境を提供してくださったFUN Minecraft Serverの運営の皆様に深くお礼申し上げます。


% TODO: 発表等実績
\chapter*{発表・採録実績}

% TODO: 発表・採録実績(確定分も含む)を以下の例のように記入

\subsection*{発表等}
\begin{enumerate}
\renewcommand{\labelenumi}{[\arabic{enumi}]}
    \item Research and Education Through a Minecraft Testbed: Character, Bots and ChatGPT,日本教育工学会 2023年秋季全国大会
    \item An LLM Chatbot in Minecraft with Educational Applications,日本教育工学会 2024年春季全国大会 発表予定(発表予定2024年3月)
\end{enumerate}



% TODO: 参考文献
\begin{thebibliography}{99}
\bibitem{bib:chess}
    J. McCarthy. (2007) 
    ``What is Artificial Intelligence?'', 
    \url{http://www-formal.stanford.edu/jmc/whatisai/whatisai.html} (Accessed 2023/10/31).
\bibitem{bib:AplpaGo}
    D. Silver, A. Huang, C. J. Maddison et al. (2016) 
    ``Mastering the game of Go with deep neural networks and tree search'', 
    Nature volume 529, pp.484-489.
\bibitem{bibAlphaStar}
    O. Vinyals, I. Babuschkin, W. M. Czarnecki et al. (2018) 
    ``Grandmaster level in StarCraft II using multi-agent reinforcement learning'', 
    Nature volume 575, pp.350-354, 2019.
\bibitem{bib:maia}
    R. McIlroy-Young, S. Sen, J. Kleinberg, A. Anderson. (2020) 
    ``Aligning Superhuman AI with Human Behavior: Chess as a Model System'',
    KDD '20: Proceedings of the 26th ACM SIGKDD International Conference on Knowledge Discovery \& Data Mining, pp 1677-1687.
\bibitem{bib:CraftAssist}
    J. Gray, K. Srinet, Y. Jernite et al. (2020)
    ``CraftAssist: A Framework for Dialogue-enabled Interactive Agents'',
    facebook research.
\bibitem{bib:Minecraft}
    Mojang Studios. (2023)
    ``Minecraft'', 
    \url{https://www.minecraft.net/ja-jp} (Accessed 2023/11/12).
\bibitem{bib:minecraft_news}
    Yahooニュース. (2023)
    ``『マインクラフト(Minecraft)』の累計売り上げが3億本を突破!史上最も売れたインディーゲーム,不動の地位を築く'',
    \url{https://news.yahoo.co.jp/articles/81edc817187b79350186eed8b5219f6be51a3ac7} (Accessed 2023/11/12).
\bibitem{bib:Voyager}
    G. Wang, Y. Xie, Y. Jiang et al. (2023)
    ``Voyager: An Open-Ended Embodied Agent with Large Language Models'', 
    arXiv preprint arXiv: Arxiv-2305.16291.
\bibitem{bib:Mineflayer}
    PrismarineJS. (2024) 
    ``Github/PrismarineJS/mineflayer'', 
    \url{https://github.com/PrismarineJS/mineflayer} (Accessed 2024/01/09).
\bibitem{bib:minecraft_edu}
    Mojang Studios, 
    ``Minecraft Education'', 
    \url{https://education.minecraft.net/ja-jp} (Accessed 2024/01/08).
\bibitem{bib:rag}
    P. Lewis, E. Perez, A. Piktus et al. (2020) 
    ``Retrieval-Augmented Generation for Knowledge-Intensive NLP Tasks'', 
    NeurIPS.
\bibitem{bib:fun_minecraft_server}
    Kabashima, S., Matsuda, K.,Yamamoto, R., and Frank, I. (2023) 
    ``Building a Better World: Lessons from Setting up, Maintaining and Developing a University Minecraft Server'',
    JSET Autumn Conference 2023 collection papers, pp.597-598.
\end{thebibliography}


% 図表一覧等自動生成
\listoffigures
\thispagestyle{plain}
\listoftables
\thispagestyle{plain}


\end{document}
