\chapter{まとめ・展望}	
\thispagestyle{plain}   % chapterの直後に必ず指定

\section{まとめ}
本研究は,人間とAIの協力・協調関係のを目的とし,Minecraftの中で人間との対話や,作業の支援を行うbot,{\mason}の作成を行った.
{\mason}はタスクの提示やプレイヤーとの会話機能,構造物の自動生成機能などを有しており,それらの動作テストを行いその結果の考察を行った.
考察からはテストレベルでの興味深い結果を得られることが判明したものの実用段階に至るにはまだいくつかの発展が必要であるという課題が示唆された.
一方で大規模言語モデルの発展が著しいため,将来的にはより高精度なモデルが使用可能,ファインチューニング可能になることで,
コミュニケーションスキルや問題解決スキル,創造的なインスピレーションをAIから得られる可能性が考えられた.

\section{展望}
{\mason}は一般的なフレームワークをGithub(\url{https://github.com/koya328/Mason})で公開予定である.
大規模言語モデルがより創造的に進化した場合,他の開発者が{\mason}を使用して可能性を探求することが可能である.
現状での{\mason}の改良点として,サバイバルモード向けの対話機能は,複数の大規模言語モデルの組み合わせにより実現できているため,カスタマイズすることで,より多彩な行動・会話ができるのではないかと考えられる.
例として,複数人のコミュニティで使用するのであれば,社会的でないプレイヤーの発言などを監視する機能や,サーバーのログを収集し,サーバーの最新の情報について質問可能な機能などが考えられる.
構造物の自動生成ではファインチューニングのデータ件数が少なかった.データを増やすことにより,空間把握能力はないものの,人間がイグルーやプラミッドなどの特殊な構造物を作る際,どのようなブロックを使うかなどの情報は学習できるのではないかと考えられる.