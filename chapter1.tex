\chapter{序論}	% TODO: 章題を記入.題は任意.
\thispagestyle{plain}   % chapterの直後に必ず指定

%TODO: 章の内容を記入.以下はサンプル.

\section{背景・目的}
本研究は変化し続ける人工知能技術の中で,人間とAIとの協調・協力関係の境界と可能性を探ることを目的としている.
目的の達成のために,Minecraft内での人間との対話や作業支援を行うBOTの開発を通じて検証を行う.

本研究の背景としては,``チェスはAIのショウジョウバエである''\cite{bib:chess}と言うように,ゲームは人工知能やアルゴリズムの性能向上のための実験台や基礎研究として古くから重要視されてきたことがあげられる.
特に2010年代に入ってからは,囲碁やスタークラフトといった複雑なゲームにおいて,AIが人間のスコアを超越するようになった\cite{bib:AplpaGo,bibAlphaStar}.
しかし,AIが人間を圧倒してもなお,ゲーム研究は実験台や基礎研究として価値のあるものであり,ここ数年では,人間とAIの協力関係と言った新たなアプローチが見られるようになっている\cite{bib:maia,bib:CraftAssist}.

よくAIの研究で使用されるゲーム環境として,``Minecraft''\cite{bib:Minecraft}が使用されている.
Minecraftは2014年に最も売れたゲームとして,ギネス世界記録に乗り,2023年には3億本を売り上げた認知度の高いゲームである\cite{bib:minecraft_news}.
また,Minecraftは,ユーザーが自由に世界を創造・探索できるゲームであり,その特性から教育や研究のツールとしても幅広く使われている.

そこで本研究では,Minecraftのその特性に焦点を当て,Minecraftの中でLLMを搭載したBOTを作成し,そのBOTと人間が対話を行ったり,共同作業を行ったりすることで,人間とAIの協力・協調関係の検証を行う.
LLMを搭載したBOTが3D空間にて人間と対話を行うという試みはほとんど存在していないため,本研究によってLLMの新たな活用法を見い出すことを期待している.
また,BOTの行った建築作業などから,人間が建築する際の新たなアイデアやインスピレーションを得られるのではないかと考えている.

\section{章構成}