\chapter{関連研究}	
\thispagestyle{plain}   % chapterの直後に必ず指定

本章では,本研究に関する,人間とAIの協調,Minecraftの教育性,自動生成,Minecraftのエージェント等の研究,文献について述べる.

\section{人間とAIの協調に関する研究}
この節では人間との協調を目的としたAIに関する関連研究について述べる.

\subsection{Maia}
Kleinbergらは,人間らしい手を打つニューラルネットワークチェスエンジンの“Maia”を開発した\cite{bib:maia}.
Maiaはオンラインでチェスを行うプレイヤーの棋譜を,強さ別に9つのレベルに分けて学習しており,レベル別にプレイヤーがどのようなチェスの指し方をしているかを学べるようになったと述べている.
また,低いレベルと高いレベルの学習結果の違いから,未熟なチェスプレイヤーがどのような間違いを犯すかを「チェス学習ツール」として予測することが可能である.
Kleinbergは,人間に教えたり助けたりするAIの研究によって,人間ともっとうまく交流できるかもしれない,あるいは交渉すらできるかもしれないと述べている.


\section{Minecraftの教育性に関する研究}
この節ではMinecraftの教育性に関する関連研究について述べる.

\subsection{学習ツールとしてのMinecraft}
Cipolloneらは高校の文学の授業でMinecraftを使用し,文学の授業を受けた生徒らはMinecraftを録画する形で映画を3本作成した\cite{bib:minecraft_creative tool}.
Cipolloneらは作成された映画をプロットとキャラクターの観点で検討したところ,Minecraftを用いたことで生徒が創造性を発揮し,概念を理解しやすくなったと述べている.
また,Minecraftは他の方法ではコストがかかったり不可能だったりするような作品を生徒が作れるようになるツールであると述べている.
しかし,ゲーム文化と,正式な学校教育の文化とではギャップが大きいため,Minecraftのような教育ゲームを導入するためには,ゲーム文化をもっと受け入れられるような考えが必要かもしれないと提案している.


\section{自動生成に関する研究}
この節では自動生成に関する関連研究について述べる.

\subsection{ARLPCG}
Gisslénらは,手続き型コンテンツ生成の新たなアプローチとして,Adversarial Reinforcement Learning for Procedural Content Generation(ARLPCG)を提案した\cite{bib:arlpcg}.
手続き型コンテンツ生成の一般的なアプローチとしては,異なる環境を手続き的に生成し,学習したエージェントの汎用性を高める方法が挙げられるが,ARLPCGでは代わりに,PCGエージェント(Generator)と攻略エージェント(Solver)からなる敵対的モデルを用いている.
GeneratorはSolverのパフォーマンスに基づいて報酬信号を受け取っており,攻略不可能ではないが挑戦的なコンテンツ生成が可能になったと述べている.
また,モデルの制御を行うためにGeneratorに補助入力を使用し,検証として3DCGのアクションゲームとレースゲームの2つを生成したところ,ARLPCGの解答率が大幅に向上し,補助入力によって難易度がある程度調整可能になったと述べている.

\subsection{GANcraft}\label{sec:gancraft}
Haoらは,Minecraftのような3Dのブロック世界からリアルな地形画像を生成する,GANcraftを開発した\cite{bib:gancraft}.
GANcraftは教師なしのニューラルレンダリングフレームワークである.
本手法では,ブロック世界を入力とし,各ブロックには土,草,木,砂,水などのラベルが割り当てられる.
また,ブロックの世界を連続的な体積関数として表現し,教師データがない場合でも,任意の視点からなリアルな地形画像をレンダリングできるようにモデルを学習できると述べている.
GANcraftは,その他のフレームワークと比較して有効性を示しており,長年の経験が必要とされる複雑な風景の3Dモデリングを簡単に行うことができると述べている.

\section{Minecraftのエージェントに関する研究}
この節ではMinecraftのエージェントに関する関連研究について述べる.

\subsection{CraftAssist}
Grayらは,Minecraftで人間のプレイヤーと協力できるAIアシスタントを実装するためのプラットフォームとして,CraftAssistを開発した\cite{bib:craft_assist}.
CraftAssistのアシスタントはブロックの配置や破壊,人間のプレイヤーとのチャットが可能であり,また,言語,知覚,記憶,身体的動作を組み合わせることで,家の建設などの複雑なタスクを実行できる.
このような研究は,実世界で人間とより良く対話し協力できるAIアシスタントを実現する手助けとなる可能性があり,AIアシスタントとの対話を通じて新しい概念とスキルを積極的に学び取ることができるかもしれないと述べている.

\subsection{Voyager}
Wangらは,Minecraftで大規模言語モデルを使用し,人間の介入なしに継続的に世界を探索するエージェントのVoyagerを開発した\cite{bib:voyager}.
VoyagerはAPIを介して自分の所持アイテムや近くの地形などのゲーム情報を読み取ることができ,GPT-4を用いて,情報をもとに短期的な目標を設定したり,目標を達成するために必要なコードを生成したりすることが可能である.
Voyagerはほかのモデルと比較して広い範囲の地形を探索し,多くのアイテムを製作することができたと述べている.
またWIRED誌では``このような方法で言語モデルを活用すれば,オフィスで多くの日常的な作業を自動化できるかもしれない.それがVoyagerがもたらした最大の経済効果であろう''と述べられている.
