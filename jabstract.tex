% TODO: 日本語アブストラクトを以下の{}内に記述(以下はダミーテキスト)
\newcommand{\jabstract}{

    ゲームは人工知能やアルゴリズムの性能向上のための実験台や基礎研究として古くから重要視されており,近年では囲碁などの複雑なゲームにおいてAIが人間のスコアを超越するようになった.
    しかし,AIが人間を圧倒してもなお,ゲーム研究は実験台や基礎研究として価値のあるものであり,人間とAIの協力関係と言った新たなアプローチが見られるようになっている.
    本研究では変化し続ける人工知能技術の中で,人間とAIとの協調・協力関係の境界と可能性を探ることを目的としており,仮想3Dゲーム「Minecraft」において,ChatGPT APIを活用してプレイヤーと連携するbot "Mason"(Minecraft AI Support - Objectives and Novelty)を作成した.
    Masonはプレイヤーとの会話やプレイヤーの情報を観察することで,現在の状況に適した次のタスクを提案することができる.また,ユーザーの指示に従ってMinecraftのfillコマンドを発行することで,ゲームプレイの世界に家や家具などの構造物を生成し,カスタマイズすることもできる.これらの機能をテストしたところ,LLMの空間把握能力や応答性能から,現状の人間とAIとの協調・協力関係の境界が示唆された.ボット機能を仮想空間に埋め込む能力は急速に拡大しており,この結果はbotを仮想空間に埋め込む活動の初期段階とみなすことが出来る.Minecraftは教育ツールとしての用途もあり,Masonの人間とAIの協調の可能性として,Masonは生徒に仮想世界の基礎を紹介したり,創造性やコミュニケーション力を鍛え,LLMに慣れ親しんでもらうのに役立つと考えられる.今後の予定としては,プロンプトの改善によってより複雑な構造物を生成や対話を可能にし,より強いインスピレーションや能力を得られるようにする必要があると考えられる.Masonフレームワークは公開しており,他の人が適応して,生成AIの発展によって可能になった創造的な展望を探求可能性も考えられる.

}