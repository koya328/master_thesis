\chapter{序論}	
\thispagestyle{plain}   % chapterの直後に必ず指定

本章では``人間とAIの協調:ChatGPT を用いたMinecraftBOT による作業支援''の研究を行うに至った背景や目的、また本稿の章構成を述べる.

\section{背景・目的}
本研究では変化し続ける人工知能技術の中で,人間とAIとの協調・協力関係の境界と可能性を探ることを目的としている.
目的の達成のために,Minecraft内での人間との対話や作業支援を行うbotの開発を通じて検証を行う.

本研究の背景としては,``チェスはAIのショウジョウバエである''\cite{bib:chess}と言うように,ゲームは人工知能やアルゴリズムの性能向上のための実験台や基礎研究として古くから重要視されてきたことがあげられる.
特に2010年代に入ってからは,囲碁やスタークラフトといった複雑なゲームにおいて,AIが人間のスコアを超越するようになった\cite{bib:AplpaGo,bibAlphaStar}.
しかし,AIが人間を圧倒してもなお,ゲーム研究は実験台や基礎研究として価値のあるものであり,ここ数年では,人間とAIの協力関係と言った新たなアプローチが見られるようになっている\cite{bib:maia,bib:craft_assist}.

よくAIの研究で使用されるゲーム環境として,``Minecraft''\cite{bib:Minecraft}が使用されている.
Minecraftは2014年に最も売れたインディーゲームとして,2023年には3億本を売り上げた認知度の高いゲームである\cite{bib:minecraft_news}.
また,Minecraftは,ユーザーが自由に世界を創造・探索できるゲームであり,その特性から教育や研究のツールとしても幅広く使われている.

そこで本研究では,Minecraftのその特性に焦点を当て,Minecraftの中で大規模言語モデルを搭載したbot、{\mason} (Minecraft AI Support - Objectives and Novelty)を作成した。{\mason}と人間が対話を行ったり,共同作業を行ったりすることで,人間とAIの協力・協調関係の検証を行う.
ボット機能を仮想空間に埋め込む能力は急速に拡大しており,対話型インターフェースの進化が検証されている.本研究はボット機能を仮想空間に埋め込むこれらの研究と同様の,初期の研究とみなすことができる.
botの行った建築作業などからは,人間が建築する際の新たなアイデアやインスピレーションを得られるのではないかと考えている.

\section{章構成}
本稿の章構成について解説する。
第2章では本研究に関する、人間とAIの協調、Minecraftの教育性、自動生成、Minecraftのエージェント等の研究、文献について述べる。
第3章では本研究のゲーム環境であるMinecraftやテスト環境のFUN Minecraft Serverについて述べる。
第4章では本章では作成したbot、{\mason}についてや,各節で{\mason}の詳細な機能について解説する.
第5章ではMasonの動作テストや構造物自動生成の実験,アンケート実施等の概要や結果について述べる.
第6章では本研究の考察について述べる。
第7章では本研究のまとめや展望について述べる。