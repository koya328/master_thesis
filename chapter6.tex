\chapter{考察}
\thispagestyle{plain}   % chapterの直後に必ず指定

\section{タスクの提示や会話機能の考察}\label{sec:gpt_res_investigation}


\section{構造物生成の考察}\label{sec:generate_investigation}
この節では,先述の\ref{sec:build_mode_generate}節及び各小節における実験結果を考察する.
\ref{sec:build_mode_generate}節では,ビルドモードによる構造物の生成が行われたが,その結果から,いくつかの特筆すべき傾向が存在した.

\ref{sec:ex4}節の実験では,ファインチューニングされたgpt-3.5-turboモデルを使用し,その結果から複雑な構造物を生成する能力が向上したことが示唆された.
しかし,その一方で問題も生じており,その一つが建物内部の構造を埋めてしまう現象である.
これは,gpt-3.5-turboが内部構造の設計を適切に行えない,という可能性を指摘している.
また,ファインチューニング前の同モデルで駅を生成した\ref{sec:ex1}節の実験でも同様の問題が見られ,チケットカウンター,ベンチ,標識等の配置が適切でない,という結果が得られている.

一方で,モデルが更新されたgpt-4やgpt-4-turboを用いた実験,\ref{sec:ex2}節および\ref{sec:ex3}節では,この問題は見られなかった.
ここでは,それらのモデルが内部空間を適切に設計し,窓を正しい位置に配置する能力を示していた.
これは,これらのモデルが空間把握能力を著しく向上させていると考えられる.

これらの結果から,gpt-3.5-turboは空間把握能力が限定的であり,このモデルをさらにファインチューニングしても,人間と同様に複雑な構造を構築する能力を獲得するのは難しいと推測することができる.
しかし,後継のgpt-4やgpt-4-turboを用いれば,より複雑な構造物の生成が可能になると期待される.
これらのファインチューニングが実現すれば,より高度な建造物生成が可能となり,人間により強い建築のインスピレーションなどを与える可能性がある.

\section{アンケートの考察}\label{sec:survey_investigation}
本節では,先述の\ref{sec:survey_result}節及び各小節における実験結果を考察する.
そう思う~そう思わないの4段階の同じ指標で回答してもらった結果,表\ref{tab:answer2}の結果のみ平均3.47と高かったことから,botによって大規模言語モデルの新しい使い方を学べたことが示唆された.
表\ref{tab:answer1}の結果では肯定的な意見が得られているものの,表\ref{tab:answer2}の結果より平均が低いことや\ref{sec:build_mode_generate}節の生成結果から考察するとビルドモードは改善の必要があると考えられる.

\section{考察}
本節では、\ref{sec:gpt_res_investigation}節~\ref{sec:survey_investigation}節の考察をもとに本研究の目的である、人間とAIの協調の限界・可能性について考察する。